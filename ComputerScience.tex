\documentclass[a4paper,10pt,twocolumn,oneside]{article}
\setlength{\columnsep}{10pt}                                                              %兩欄模式的間距
\setlength{\columnseprule}{0pt}                                                                %兩欄模式間格線粗細

\usepackage{amsthm}								%定義,例題
\usepackage{amssymb}
%\usepackage[margin=2cm]{geometry}
\usepackage{fontspec}								%設定字體
\usepackage{color}
% \usepackage{float}
%\usepackage{subfigure}
\usepackage[x11names]{xcolor}
\usepackage{listings}								%顯示code用的
%\usepackage[Glenn]{fncychap}						%排版,頁面模板
\usepackage{fancyhdr}								%設定頁首頁尾
\usepackage{graphicx}								%Graphic
\usepackage{enumerate}
\usepackage{float}
\usepackage{titlesec}
\usepackage{amsmath}
\usepackage[CheckSingle, CJKmath]{xeCJK}
% \usepackage{CJKulem}
%\usepackage[T1]{fontenc}
\titlespacing{\section}{0cm}{0cm}{0cm}
\titlespacing{\subsection}{0cm}{0cm}{0cm}
\usepackage{amsmath, courier, listings, fancyhdr, graphicx}
\topmargin=0pt
\headsep=5pt
\textheight=780pt
\footskip=0pt
\voffset=-40pt
\textwidth=545pt
\marginparsep=0pt
\marginparwidth=0pt
\marginparpush=0pt
\oddsidemargin=0pt
\evensidemargin=0pt
\hoffset=-42pt

%\renewcommand\listfigurename{圖目錄}
%\renewcommand\listtablename{表目錄} 

%%%%%%%%%%%%%%%%%%%%%%%%%%%%%

\setmainfont{Consolas}				%主要字型
\setmonofont{Monaco}				%主要字型
\setCJKmainfont{Noto Sans TC}
%\setCJKmainfont{Consolas}			%中文字型
%\setmainfont{sourcecodepro}
\XeTeXlinebreaklocale "zh"						%中文自動換行
\XeTeXlinebreakskip = 0pt plus 1pt				%設定段落之間的距離
\setcounter{secnumdepth}{3}						%目錄顯示第三層

%%%%%%%%%%%%%%%%%%%%%%%%%%%%%
\makeatletter
\lst@CCPutMacro\lst@ProcessOther {"2D}{\lst@ttfamily{-{}}{-{}}}
\@empty\z@\@empty
\makeatother
\lstset{											% Code顯示
language=C++,										% the language of the code
basicstyle=\footnotesize\ttfamily, 						% the size of the fonts that are used for the code
%numbers=left,										% where to put the line-numbers
numberstyle=\footnotesize,						% the size of the fonts that are used for the line-numbers
stepnumber=1,										% the step between two line-numbers. If it's 1, each line  will be numbered
numbersep=5pt,										% how far the line-numbers are from the code
backgroundcolor=\color{white},					% choose the background color. You must add \usepackage{color}
showspaces=false,									% show spaces adding particular underscores
showstringspaces=false,							% underline spaces within strings
showtabs=false,									% show tabs within strings adding particular underscores
frame=false,											% adds a frame around the code
tabsize=2,											% sets default tabsize to 2 spaces
captionpos=b,										% sets the caption-position to bottom
breaklines=true,									% sets automatic line breaking
breakatwhitespace=false,							% sets if automatic breaks should only happen at whitespace
escapeinside={\%*}{*)},							% if you want to add a comment within your code
morekeywords={*},									% if you want to add more keywords to the set
keywordstyle=\bfseries\color{Blue1},
commentstyle=\itshape\color{Red4},
stringstyle=\itshape\color{Green4},
}
\begin{document}

%%%%%%%%%%%%%%%%%%%%%%%%%%%%%%%%%%%%%%%
\pagestyle{fancy}
\fancyfoot{}
%\fancyfoot[R]{\includegraphics[width=20pt]{ironwood.jpg}}
\lhead{\fontsize{9pt}{9pt}\selectfont National Taiwan Ocean University Computer Science Mid-term Exam Translation}
\rhead{\fontsize{9pt}{9pt}\selectfont Written By Mr.JB \& Brass}
% \fancyhead[L]{National Taiwan Ocean University Computer Science Mid-term Exam Translation}
% \fancyhead[R]{Written By Mr.JB \& Brass}
\renewcommand{\headrulewidth}{0.4pt}
%\renewcommand{\contentsname}{Contents} 
\renewcommand{\footnotesize}{3pt}%设置脚注字体大小
\scriptsize
%\tableofcontents
%%%%%%%%%%%%%%%%%%%%%%%%%%%%%%%%%%%%%%%%%
\setlength{\parindent}{0pt}
\begin{normalsize}
    
    \begin{table}[!]
        \section{Data Storage}
        \centering
        \begin{tabular}{|c|c|}
            \hline
            VLSI&超大型積體電路\\
            \hline
            stream&串流\\
            \hline
            hexadecimal&十六進位\\
            \hline 
            RAM&隨機存取記憶體\\
            \hline 
            DRAM\SDRAM&動態記憶體\\
            \hline
            track&磁軌\\
            \hline
            cylinder&磁柱\\
            \hline
            sector&磁區\\
            \hline
            zone-bit recording&分區位元儲存法\\
            \hline 
            seek time&搜尋時間\\
            \hline
            rotational delay&旋轉延遲\\
            \hline
            access time&存取時間\\
            \hline
            transfer rate&傳輸速度\\
            \hline
            bandwidth&頻寬\\
            \hline
            latency&延遲時間\\
            \hline
            magnetic tape&磁帶機\\
            \hline
            floppy disk drives&軟碟機\\
            \hline
            computer-aided design&電腦輔助設計\\
            \hline
            raidx point&基數點\\
            \hline
            temporal masking&時域遮罩\\
            \hline
            frequency masking&頻域遮罩\\
            \hline
        \end{tabular}    
    \end{table}
    
    \begin{table}[!]
        \section{Data Manipulation}
        \centering
        \begin{tabular}{|c|c|}
            \hline
            arithmetic/logic unit&邏輯元件\\
            \hline
            control unit&控制元件\\
            \hline
            register unit&暫存器元件\\
            \hline
            cache memory&快取記憶體\\
            \hline
            RISC&精簡指令集計算機\\
            \hline
            CISC&複雜指令集計算機\\
            \hline
            benchmarkig&評量基準\\
            \hline
            circular shift&循環位移\\
            \hline
            rotation&循環\\
            \hline
            logical shift&邏輯位移\\
            \hline
            arithmetic shifts&算數位移\\
            \hline
            parallel communication&平行通訊\\
            \hline
            serial communication&串列通訊\\
            \hline
            modem&數據機\\
            \hline
            broadband&寬頻\\
            \hline
        \end{tabular}   
    \end{table}
    
    \begin{table}[!]
        \section{Operating System}
        \centering
        \begin{tabular}{|c|c|}    
            \hline
            batch processing&批次處理\\
            \hline
            interactive processing&交談式處理\\
            \hline
            real-time processing&及時處理\\
            \hline
            multiprogramming&多程式化\\
            \hline
            multitasking&多工\\
            \hline
            load balancing&負載平衡\\
            \hline
            scaling&分割\\
            \hline
            embedded systems&嵌入式系統\\
            \hline
            utility software&公用軟體\\
            \hline
            kernel&核心\\
            \hline
            directory path&目錄路徑\\
            \hline
            device drivers&裝置驅動程式\\
            \hline
            process table&處理程序表\\
            \hline
            context switch&數據交換\\
            \hline
            interrupt handler&中斷處置器\\
            \hline
            set&(旗標狀態)有\\
            \hline
            clear&(旗標狀態)無\\
            \hline
            interrupt disable&中斷去能\\
            \hline
            interrupt enable&中斷賦能\\
            \hline
            semaphore&旗號\\
            \hline
            mutual exclusion&互斥\\
            \hline
            deadlock&死結\\
            \hline
            forking&分叉\\
            \hline
            spooling&排存\\
            \hline
            administrator&管理者\\
            \hline
            auditing software&多稽核軟體\\
            \hline
            sniffing software&監聽軟體\\
            \hline
            privileged instruction&特權指令\\
            \hline
        \end{tabular}  
    \end{table}
    
    \begin{table}[!]
        \section{Network and Internet}
        \centering
        \begin{tabular}{|c|c|}
            \hline
            personal area network&個人區域網路\\
            \hline
            local area network&區域網路\\
            \hline
            metropolitan area networ&都會區域網路\\
            \hline
            proprietary network&專屬網路\\
            \hline
            acess point(AP)&存取點\\
            \hline
            hub&集線器\\
            \hline
            protocols&協定\\
            \hline
            CSNA/CD&載波偵測多重存取碰撞偵測\\
            \hline
            CSMA/CA&載波偵測多重存取碰撞避免\\
            \hline
            Distributed Coordination&\\
            Function/DCF&分散式協調功能\\
            \hline
            Contention& \\ 
            Service&競爭式服務\\
            \hline
            Point Coordination& \\
            Function/PCF&集中式協調功能\\
            \hline
            Contention-Free Service&無競爭式服務\\
            \hline
            repeater&中繼器\\
            \hline
            bridge&橋接器\\
            \hline
            switch&交換機\\
            \hline
            routers&路由器\\
            \hline
            forwarding table&轉送表\\
            \hline
            gateway&閘道\\
            \hline
            client/server&主從模式\\
            \hline
            peer-to-peer/P2P&點對點\\
            \hline
            swarm&群體\\
            \hline
            distributed systems&分散式系統\\
            \hline
            cluster computing&叢集計算\\
            \hline
            grid computing &網格運算\\
            \hline
            elastic compute cloud&彈性雲端運算\\
            \hline
        \end{tabular}
    \end{table}
\end{document}